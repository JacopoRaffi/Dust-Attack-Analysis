\chapter*{Ringraziamenti} % L'asterisco permette di non indicizzare il capitolo (e quindi non gli da un numero)
\begin{flushright}
\itshape 
Ringrazio la mia famiglia, in particolare i miei genitori e mio fratello, per avermi sostenuto nei momenti più bui della mia vita\\
Ringrazio Sara non solo per avermi sostenuto nei miei momenti difficili ma anche per avermi corretto lo stile delle relazioni scritte durante questi tre anni di università\\ 
Vorrei inoltre ringraziare i miei amici universitari Simone, Giacomo e Dario per la loro compagnia, per il loro aiuto, per tutte le risate che abbiamo fatto insieme e in generale per avermi fatto trascorrere dei bellissimi momenti durante questi tre anni di università.\\
Infine ci tengo a ringraziare i miei due relatori, la prof.ssa Ricci e il prof. Di Francesco, il Dott. Loporchio e la prof.ssa Bernasconi per avermi permesso di svolgere la tesi con loro e per tutta l'immensa pazienza e disponibilità che hanno avuto nei miei confronti.
\end{flushright}


\chapter{Introduzione}
Se ripercorriamo la storia dell'uomo, possiamo osservare come i mezzi per lo scambio di beni tra le persone sia profondamente mutato nel corso del tempo, in relazione alla trasformazione culturale e tecnologica cui é andata incontro l’umanità: dal baratto siamo passati alle monete d’oro, fino ad arrivare, all’immaterialità di assegni o carte di credito. Il concetto di denaro stesso, nel suo significato più generale di mezzo per consentire lo scambio di valore, non é sfuggito alla diffusione che Internet ha avuto, portando alla nascita di una tipologia di valuta completamente nuova denominata \textbf{criptovaluta} o \textbf{moneta elettronica}. 

Negli ultimi anni si sono sviluppate vari tipi di criptovalute, ognuna con i propri protocolli e le proprie specifiche tecniche, però la prima moneta elettronica sviluppata e che ha acquisito maggior valore nel tempo é \textbf{Bitcoin}. 

Bitcoin nasce con l'intento di creare una moneta completamente libera da controlli di tipo governativo o bancario, consentendo agli utenti di effettuare transazioni attraverso una rete decentralizzata secondo il modello Peer-to-Peer. Quindi la validazione della correttezza delle transazioni non é in mano ad una banca o ad un ente di terze parti, ma é nelle mani di tutti gli utenti della rete. 

Un’altra parola chiave per comprendere il funzionamento di Bitcoin é l’anonimato: Bitcoin é considerato un sistema anonimo, ma sarebbe corretto utilizzare l’espressione \textbf{pseudo-anonimo}.

Ogni utente che partecipa alla rete Bitcoin infatti viene identificato non dal proprio nome o cognome, bensì da un address, che nulla lascia trasparire sulla reale identità dell'utente che vi sta dietro. Un utente inoltre può utilizzare address differenti ogni volta che esegue
una transazione, rendendo in pratica complicato capire che dietro ad un insieme di address ci sia in realtà il medesimo utente. Il fatto che ogni utente possa generare le proprie transazioni porta ad una serie di problemi che non sono presenti nei sistemi di scambio tradizionali. Il problema più noto viene definito ''Double Spending". Prevenire il double spending significa impedire che un utente possa spendere più volte lo stesso importo in transazioni differenti, questo problema non esiste nei sistemi tradizionali perché lo scambio di valore avviene tramite ente centralizzato che ha il controllo dei fondi degli utenti. 

La soluzione adottata da Bitcoin é quella di permettere agli utenti di conoscere le transazioni validate precedentemente, quindi l'intero storico delle transazioni é reso pubblico a chiunque.

Il libro contabile elettronico contenente tutte le transazioni é implementato tramite una struttura dati completamente pubblica e immutabile: la blockchain, all’interno della quale sono registrati tutti i movimenti di Bitcoin a partire dalla prima transazione, fino ai giorni nostri. Nella blockchain quindi é possibile osservare tutte le transazioni eseguite dai vari address di Bitcoin, questo porta a un serio problema legato alla privacy degli utenti. 

Nel corso degli anni sono stati sviluppati diversi attacchi in grado di violare l'anonimato di Bitcoin, la tipologia di attacchi più studiata é quella basata sull'analisi della blockchain. L'obiettivo di questi attacchi é la violazione dell'anonimato di un utente aggregando diversi address in cluster, ciascuno associato ad un singolo utente; questo avviene tramite l'analisi della blockchain e l'utilizzo di particolari regole euristiche.

Lo scopo di questa tesi é l'analisi di un attacco di de-anonimizazzione definito \textbf{Dust Attack}. Il Dust Attack basa la propria strategia sull'invio del dust, una piccola quantità di criptovaluta presente in un UTXO sotto i limiti minimi di scambio, per questo motivo é necessario aggregare un importo dust insieme ad altri importi. L'euristica su cui si basa il Dust Attack afferma che tutti gli address di input di una transazione appartengono allo stesso utente, l'obiettivo dell'attaccante quindi é inviare il dust affinché venga speso insieme ad altri input in modo da poter collegare tutti gli address e formare un unico cluster da associare ad un singolo utente. In questo modo non solo l'attaccante può tracciare l'attività di un utente ma se scopre l'identità del proprietario di uno di questi address automaticamente scopre che tutti gli address del cluster appartengono al medesimo utente. Dopo aver scoperto l'identità del proprietario é possibile eseguire elaborati attachi di phishing atti a rubare le chiavi private degli address così da poter rubare i fondi di quell'utente.

Le analisi vertono in particolare sull'efficacia che possa avere il Dust Attack, in particolare verranno mostrate delle statistiche atte a dimostrare se e quanto sia stato speso il dust e se questi importi abbiano permesso una qualche de-anonimizzazione, infine verranno descritti alcuni pattern che potrebbero rappresentare dei possibili Dust Attack.

Segue questa introduzione il capitolo 2, nel quale verrà descritta la tecnologia alla base di Bitcoin. L'attenzione sarà posta sulla blockchain, sul funzionamento delle transazioni e il relativo protocollo di pagamento, verrà spiegato in breve l'UTXO set e infine verrà descritto il problema relativo all'anonimato di Bitcoin mostrando alcune euristiche utilizzate per la de-anonimizzazione degli address.

Nel capitolo 3 verrà spiegato più nel dettaglio il significato di dust e i suoi possibili utilizzi. Verrà descritto in dettaglio il Dust Attack, spiegando il suo funzionamento, le sue conseguenze e le possibili contromisure.

Nel capitolo 4 invece verrà mostrata la formattazione dei dati e verranno descritti alcuni algoritmi realizzati per l'analisi dei dati.

Il capitolo 5 mostrerà le statistiche realizzate. Inizialmente verranno mostrati i risultati ottenuti dopo aver applicato vari filtraggi del dataset. Successivamente verranno analizzate le distribuzioni degli input e degli output dust, in particolare la distribuzione del dust generato negli anni. In seguito gli importi dust sono stati classificati per mostrare come siano stati spesi negli anni, in particolare per dimostrare che il dust abbia permesso una de-anonimizzazione degli address. Dopo aver mostrato come diverse transazioni abbiano più di due address di input diversi viene mostrata la distribuzione del numero di address diversi all'interno delle transazioni con almeno due address differenti. Infine verranno descritti alcuni pattern di possibili Dust Attack.